\documentclass[noamsthm]{beamer}
% ,handout,notes=onlyslideswithnotes

\usepackage[center]{tiddemath}
\usepackage{tiddetext}
\usepackage{tiddedae}
\usepackage{parskip}

\usepackage[utf-8]{inputenc}
\usepackage[T1]{fontenc}

%\usepackage{beamerthemesplit}
\usetheme{JuanLesPins}

\usepackage{verbatim}

\title{Shapes}
\subtitle{--- ett hyfsat funktionellt ritspråk}
\author{Henrik Tidefelt}
\institute{LiTH}
\date{\today}

\newcommand{\maincoveringmode}{invisible}
\usepackage[swedish]{babel}

\providecommand{\paragraph}[1]{\textbf{#1}\,}

\newcounter{tmppauses}

\begin{document}

% svenska är förvalt, men det här visar hur man skulle gjort för att byta till det andra språket:
\selectlanguage{swedish}

\frame{%
\note[item]{Hälsa alla välkomna!}%
\note[item]{Presentera mig.}
\note[item]{Mycket hellre frågor under tiden än efteråt.}
\titlepage}

\section*{Översikt}
\frame{\frametitle{Mål}
  Med den här UppLYSningen hoppas jag
  \pause
  \begin{itemize}[<+->]
  \item Att ni ska få ett hum om vad \Shapes är.
  \item Få höra era invändningar mot designen som den ser ut idag.
  \item Lyckas hitta någon testpilot.
  \item Väcka intresse för utvecklingssamarbete.
  \end{itemize}
  \note[item]{Inget att säga här.}
}

\frame{\frametitle{Plan}
  De stora inslagen idag är:
  \pause
  \begin{itemize}[<+->]
  \item Beskriva hur språkets struktur ser ut idag.
  \item Visa lite av de funktioner som kärnan erbjuder.
  \item Diskutera intressanta utmaningar för framtiden.
  \end{itemize}
  \note[item]{Påminn om att man gärna får ställa frågor!}
}

\setbeamercovered{\maincoveringmode}

\section{Introduktion}
\frame{%
  \begin{center}
    {\Huge Introduktion}
  \end{center}
  \note{Då börjar vi\ldots}
}

\frame{\frametitle{Rötter}
  \Shapes har sina rötter i många av de språk jag varit i kontakt med:
  \begin{itemize}[<+->]
  \item MetaPost (en omarbetning av Knuths MetaFont) --- \Shapes kom till när jag ledsnade på MetaPost.
  \item Scheme --- syntax och funktions-begreppet.
  \item Haskell --- för sina rena ideal.
  \item C++ --- utmatningssyntaxen.
  \end{itemize}
  \note[item]{Inget att säga här.}
}
\frame{\frametitle{Alternativ}
  Några andra ritspråk som finns och/eller används idag:
  \begin{itemize}[<+->]
  \item MetaPost
  \item Asymptote
  \item PGF och TikZ
  \item Haskell~PDF
  \end{itemize}
  \note[item]{Inget att säga här.}
}
\frame{\frametitle{Varför \Shapes?}
  Givet utbudet av alternativa rit-språk, varför utveckla ett till?  Här är några skäl:
  \begin{itemize}[<+->]
  \item Inte funktionellt orienterade (alla utom Haskell~PDF).
  \item Dålig beräkningskapacitet (MetaPost och PDF/TikZ).
  \item Saknar domän-specifik syntax (Haskell~PDF).
  \item Inte publicerade när \Shapes påbörjades (Asymptote och Haskell~PDF).
  \end{itemize}
  \note[item]{Inget att säga här.}
}

\frame{\frametitle{Hello, shaper!}

\texttt{
$\bullet$page <{}< [stroke (0cm,0cm)-{}-(1cm,1cm)]
}

\pause

\texttt{
$\bullet$page <{}< stroke [] ((0cm,0cm)-{}-(1cm,1cm))
}

\pause

\texttt{
[($\backslash$ $\bullet$dst pth .> \{ $\bullet$dst <{}< stroke [] pth \} )\\
\ \ $\bullet$page \ (0cm,0cm)-{}-(1cm,1cm)]
}

  \note[item]{Kör live-demo!}
}

\section{Språkets struktur}
\frame{%
  \begin{center}
    {\Huge Språkets struktur}
  \end{center}
  \note[item]{Inget att säga här.}
}

\frame{\frametitle{Bindningar och scopes}
  \note[item]{Inget att säga här.}
}
\frame{\frametitle{Funktionsanrop}
  \note[item]{Inget att säga här.}
}
\frame{\frametitle{Kontrollerade tillstånd}
  \note[item]{Inget att säga här.}
}
\frame{\frametitle{Structures}
  \note[item]{Inget att säga här.}
}
\frame{\frametitle{Lat evaluering}
  \note[item]{Inget att säga här.}
}
\frame{\frametitle{Continuation passing style}
  \note[item]{Inget att säga här.}
}


\section{Funktioner i kärnan}
\frame{%
  \begin{center}
    {\Huge Funktioner i kärnan}
  \end{center}
  \note[item]{Inget att säga här.}
}

\frame{\frametitle{Kurv-konstruktion}
  \note[item]{Inget att säga här.}
}
\frame{\frametitle{Grundläggande ritning}
  \note[item]{Inget att säga här.}
}
\frame{\frametitle{2D}
  \note[item]{Inget att säga här.}
}
\frame{\frametitle{3D}
  \note[item]{Inget att säga här.}
}
\frame{\frametitle{PDF}
  \note[item]{Inget att säga här.}
}
\frame{\frametitle{\LaTeX{} och strängar}
  \note[item]{Inget att säga här.}
}

\section{Utmaningar för framtiden}
\frame{%
  \begin{center}
    {\Huge Utmaningar för framtiden}
  \end{center}
  \note[item]{Inget att säga här.}
}

\frame{\frametitle{Trixelering}
  \note[item]{Inget att säga här.}
}
\frame{\frametitle{Kompilera funktioner till PDF}
  \note[item]{Inget att säga här.}
}
\frame{\frametitle{Mer grund-struktur}
  \begin{itemize}
  \item Namespaces/packages
  \item Användar-typer
  \end{itemize}
  \note[item]{Inget att säga här.}
}

\section{Sammanfattning}
\frame{%
  \begin{center}
    {\Huge Sammanfattning}
  \end{center}
  \note[item]{Inget att säga här.}
}
\frame{\frametitle{Sammanfattning}
  \begin{itemize}
    \item \Shapes är\ldots
  \end{itemize}
  \note[item]{Inget att säga här.}
}

\frame{%
  \begin{center}
    Slut.
  \end{center}
  \note[item]{Tacka för uppmärksamheten!}
}

\end{document}
